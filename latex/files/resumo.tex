\setlength{\absparsep}{18pt} 
\begin{resumo}[Resumo]
 
Temas relacionados à análise e síntese de imagens vêm
tendo um crescente aumento de interesse nos últimos anos,
tanto por parte de pesquisadores, quanto pelo público geral.
Isso se deve ao aumento do poder computacional e da
geração de dados, que possibilitam o estudo e desenvolvimento
de modelos mais ricos, permitindo assim aplicações como 
\textit{inpainting} e transferência de estilo.

Neste trabalho será tratada a geração de texturas de
tamanho arbitrário usando uma amostra limitada,
tentando assim modelar seu processo de geração para
produzir um resultado perceptualmente semelhante
ao original. Será feita uma revisão da literatura
sobre o tema, mostrando os principais resultados e
abordagens, como a área foi se desenvolvendo até
chegar no conhecimento de hoje, e como essa área
influenciou em outros temas relacionados a imagem.

No final será mostrada uma implementação do método
computacional, aplicando-o
em diferentes tipos texturas para observar como
características da imagem original podem mudar a
qualidade do resultado. Em seguida serão exploradas
variações do método que permitem um melhor controle
da forma do resultado final.

 %Segundo a  o resumo deve ressaltar o
 %objetivo, o método, os resultados e as conclusões do documento. A ordem e a extensão
 %destes itens dependem do tipo de resumo (informativo ou indicativo) e do
 %tratamento que cada item recebe no documento original. O resumo deve ser
 %precedido da referência do documento, com exceção do resumo inserido no
 %próprio documento. (\ldots) As palavras-chave devem figurar logo abaixo do
 %resumo, antecedidas da expressão Palavras-chave:, separadas entre si por
 %ponto e finalizadas também por ponto. Deve ser redigido na terceira
 %pessoa do singular e quanto a sua extensão, o resumo deve ter de 150 a 500
 %palavras.

 Palavras-chave: síntese de textura, aprendizado profundo, transferência de estilo.
\end{resumo}

\begin{resumo}[Abstract]
 \begin{otherlanguage*}{english}

Recent years have seen an increasing interest in topics related to the analysis and synthesis of images, both from researchers and the public. This is due to the constant increasing in computational power and data generation, which make the study and development of richer models possible,
thus allowing applications such as inpainting and style transfer.

This dissertation will deal with the generation of textures of arbitrary size using a limited sample, thus trying to model its generation process and produce a result perceptually similar to the original. A review of the literature on the subject will be carried out, showing the main results and approaches, how the area has developed until it reaches today's knowledge, and how this area has influenced other topics related to image.

At the end, an implementation of the computational method will be shown, applying it to different types of textures to observe how characteristics of the original image can affect the result's quality. Next, variations of the method will be explored that allow more control over the shape of the final result.

 \end{otherlanguage*}

 Keywords: texture synthesis, deep learning, style transfer.
\end{resumo}