\setlength{\absparsep}{18pt} 
\begin{resumo}[Resumo]
 
Temas relacionados à análise e síntese de imagens vêm
tendo um crescente aumento de interesse nos últimos anos,
tanto por parte de pesquisadores, quanto pelo público geral.
Isso se deve ao aumento do poder computacional e da
geração de dados, que possibilitam o estudo e desenvolvimento
de modelos mais ricos e com melhores resultados.

Neste trabalho será tratada a geração de texturas de
tamanho arbitrário usando uma amostra limitada,
tentando assim modelar seu processo de geração para
produzir um resultado perceptualmente semelhante
ao original. Será feita uma revisão da literatura
sobre o tema, mostrando os principais resultados e
abordagens, como a área foi se desenvolvendo até
chegar no conhecimento de hoje, e como esse área
influenciou em outros temas relacionados a imagem.

No final será mostrada uma implementação do método
computacional, aplicando-o
em diferentes tipos texturas para observar como
características da imagem original podem mudar a
qualidade do resultado. Em seguida serão exploradas
variações do método que permitem um melhor controle
da forma do resultado final.

 %Segundo a  o resumo deve ressaltar o
 %objetivo, o método, os resultados e as conclusões do documento. A ordem e a extensão
 %destes itens dependem do tipo de resumo (informativo ou indicativo) e do
 %tratamento que cada item recebe no documento original. O resumo deve ser
 %precedido da referência do documento, com exceção do resumo inserido no
 %próprio documento. (\ldots) As palavras-chave devem figurar logo abaixo do
 %resumo, antecedidas da expressão Palavras-chave:, separadas entre si por
 %ponto e finalizadas também por ponto. Deve ser redigido na terceira
 %pessoa do singular e quanto a sua extensão, o resumo deve ter de 150 a 500
 %palavras.

 Palavras-chave: síntese de textura.
\end{resumo}

%\begin{resumo}[Abstract]
% \begin{otherlanguage*}{english}
%  É a tradução do resumo para o inglês (Abstract), com a finalidade de facilitar a
%  divulgação do trabalho em nível internacional.
% \end{otherlanguage*}
%
% Keywords: latex. abntex. editoração de texto.
%\end{resumo}