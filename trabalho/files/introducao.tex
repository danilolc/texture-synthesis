\chapter{Introdução}

% Falar da história

...




% Falar dos surveys

O trabalho foi baseado em dois Surveys
na área. O primeiro de Wei, Lefebvre, Kwatra
e Turk \cite{Wei2009} dá ênfase em
abordagens não paramétricas e texturas
dinâmicas. Já o segundo de Raad, Davy, 
Desolneux e Morel \cite{Raad2018} mostra
a síntese com métodos mais recentes
usando Deep Learning e Redes Convolucionais.
Analisando os dois Surveys, fica bem claro o 
quanto a área se modificou depois da
revolução do Deep Learning, que se iniciou
em 2015 com o trabalho de Yann LeCun, 
Yoshua Bengio e Geoffrey Hinton \cite{LeCun2015}.

\section{O que é textura}

% Falar do problema de definir textura
% - Na vida real
% - Em computação gráfica - imagem (conjunto de pixels)

% Definir textura como as características
% da superfície de um objeto

A palavra ``textura'' pode ter diferentes significados
dependendo do contexto.
Um deles se refere às
diferentes características da superfície de um objeto,
sejam elas visuais (cor, desenhos), geométricas (relevo,
forma) ou táteis (maciez, dureza). 
Em computação gráfica, textura geralmente é
o nome que se dá a uma imagem (matriz de pixels)
que descreve alguma
característica da superfície de um objeto,
como a cor ou a direção do vetor normal (utilizada
para iluminação).

% Restringir para texturas estacionárias
% representadas por imagens

Para o processo de síntese, é preciso restringir o
conjunto de imagens consideradas texturas àquelas
que apresentam algum tipo de padrão perceptual
em seu domínio. Com isso é possível fazer a síntese
estudando e imitando o processo que gerou esse padrão.

% Falar da definição com Markov Chain

Cross, G.R. e Jain, A.K. \cite{Cross1983} descrevem
textura como sendo um Campo Markiviano Aleatório
(Markov Random Field, MRF). Esse modelo é o mais
usado no processo de síntese pois satisfaz a propriedade
de Markov (o valor de cada pixel dado sua vizinhança não
depende do resto da textura) e a homogeneidade (a
distribuição e invariante por translação), logo
se encaixa com as necessidades descritas anteriormente.



% Mostrar imagens de texturas

 

\section{O que é síntese de textura}

% Processo de gerar uma textura "nova"
% com mesma informação perceptual

O processo de síntese de textura baseado
em amostra não tem uma definição clara
matematicamente, é algo mais intrínseco
à percepção humana. O objetivo é,
a partir de uma amostra de textura,
gerar outras texturas de tamanho
arbitrário que imitam o processo
gerador da amostra. Esse processo
gerador é baseado em métricas
perceptuais, que não podem ser
definidas de forma fechada, pois
podem depender de aspectos
finos da imagem, como forma e 
iluminação.
Assim, os trabalhos na área
nos últimos anos consistem em
tentar descobrir melhores aproximações
para essa métrica perceptual.

% Inserir imagens de exemplo

% Pode ser interpretado como uma
% reamostragem da distribuição das texturas

Ao restringir o conjunto de imagens aos
Campos Markovianos, o processo de síntese
pode ser descrito como uma re-amostragem
da distribuição condicional da amostra.
Com isso, o desafio do método passa a ser 
descobrir a distribuição a partir
da amostra.


\chapter{Modelos}

\section{Modelos paramétricos}

Os modelos que fazem a síntese a partir
de um conjunto de estatísticas da amostra
original são chamados modelos paramétricos.
Esses modelos partem de um ruido e fazem
a síntese reduzindo
a diferença entre as estatísticas desse
ruído e da amostra utilizando uma função de 
otimização.
A qualidade do modelo vai depender do
conjunto de estatísticas escolhido
e do tipo de otimização utilizado.

...

% Zhu et al [12] (MRF - Gibbs sampling)
% Heeger and Bergen [6] (matching histograms)
% De Bonet [1] (multi-resolution)
% Simoncelli and Portilla [9, 11] (wavelets)?

% Julesz’ hypothesis
% two images are perceptually equivalent if and 
% only if they agree on a set of statistic measurements.

% Falar da dificuldade em definir uma 
% estatística da informao perceptual


\section{Modelos não paramétricos}

% Falar dos modelos que extraem informação
% diretamente da imagem

Diferentemente dos modelos paramétricos,
os modelos não paramétricos não dependem
do cálculo de alguma
estatística da amostra original para
o processo de síntese. Ele gera a imagem
pegando informação diretamente da
amostra de modo a simular a amostragem
da textura.

Essa forma de amostragem foi proposta
inicialmente por Efros e Leung \cite{Efros1999}.
Ela consistia em re-amostrar a textura
pixel por pixel, pegando diretamente
da imagem original o pixel que tem
a vizinhança mais parecida com a vizinhança
de seu destino. Os resultados na época
foram bem superiores aos que se podiam
obter com os métodos paramétricos,
mas a procura por todas as vizinhanças
na amostra tornava o método lento.

Mais tarde, Efros e Freeman \cite{Efros2001}
propuseram o método de Quilting (costura),
que fazia a amostragem a partir
de pedaços da imagem original. 
O método consistia em selecionar
janelas de tamanho fixo da amostra
e distribuí-las com uma sobreposição
sobre elas. Em seguida é usado
um algoritmo de min-cut para dividir
essas janelas em pedaços que se encaixam
para formar a nova textura.
Essa abordagem era mais rápida 
computacionalmente do que o método
anterior, e produzia resultados
tão bons quanto.

Com os avanços na área,
Vivek Kwatra \cite{Kwatra2005} propôs
uma amostragem fazendo a minimização
do que ele define como função de energia. 
Essa função é a diferença quadrática
entre as vizinhanças da textura gerada
e as vizinhanças mais próximas de cada uma.
O método usa uma variação do algoritmo EM, 
onde na faze ``E'' a energia é diminuída
por mínimos quadrados, e na faze ``M'' a
energia é diminuída escolhendo as
vizinhanças mais próximas na amostra.



\section{Modelos de aprendizado profundo}

% Portilla e Simonselli
% Gatys

% Falar de multiresolução? (De Bonet)

