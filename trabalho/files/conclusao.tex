\chapter{Resultados}

% Falar da implementação (PyTorch)

Foi feita a implementação do
método de Síntese de Textura
usando a abordagem das matrizes
de Gram. 
Para isso foi utilizada 
a linguagem Python com a
biblioteca PyTorch.
O PyTorch reúne uma série de
funções que facilitaram
a implementação do método.
As principais foram: implementação
nativa da rede VGG-19, operações
aritméticas aceleradas na GPU,
cálculo automático de gradiente,
e implementação de métodos de
otimização.

% Escolha de textura

A primeira ideia era escolher um
conjunto de textura de resolução
$128 \times 128$ e tentar gerar
imagens de $256 \times 256$ pixels.
Foram escolhidos um conjunto de
diferentes tipos de texturas 
para explorar o comportamento
do algoritmo:


% Mostrar os resultados

% Falar do padding e do avgpool

% Textura em movimento

% Style transfer

% Não fica restrito a textura

% Multiescala?

\chapter{Conclusão}


% A principal vantagem de
% Redes neurais é aprender
% métricas perceptuais

% Redes neurais pre-treinadas
% não são apenas para classificação

% PyTorch é uma boa ferramenta
% de deep learning, GPU acelera


%\section{Perspectivas}
% Definir depois